\documentclass{beamer}
\usetheme{tokitex}

\usepackage{tikz}
\usepackage{graphics}
\usepackage{multirow}
\usepackage{tabto}
\usepackage{xspace}
\usepackage{amsmath}
\usepackage{hyperref}

\usepackage{tikz}
\usepackage{clrscode3e}

\usepackage[english,bahasa]{babel}
\newtranslation[to=bahasa]{Section}{Bagian}
\newtranslation[to=bahasa]{Subsection}{Subbagian}

\usepackage{listings, lstautogobble}
\usepackage{color}

\definecolor{dkgreen}{rgb}{0,0.6,0}
\definecolor{gray}{rgb}{0.5,0.5,0.5}
\definecolor{mauve}{rgb}{0.58,0,0.82}

\lstset{frame=tb,
  language=pascal,
  aboveskip=1mm,
  belowskip=1mm,
  showstringspaces=false,
  columns=fullflexible,
  keepspaces=true,
  basicstyle={\small\ttfamily},
  numbers=none,
  numberstyle=\tiny\color{gray},
  keywordstyle=\color{blue},
  commentstyle=\color{dkgreen},
  stringstyle=\color{mauve},
  breaklines=true,
  breakatwhitespace=true,
  autogobble=true
}

\usepackage{caption}
\captionsetup[figure]{labelformat=empty}

\newcommand{\progTerm}[1]{\textbf{#1}}
\newcommand{\foreignTerm}[1]{\textit{#1}}
\newcommand{\newTerm}[1]{\alert{\textbf{#1}}}
\newcommand{\emp}[1]{\alert{#1}}
\newcommand{\statement}[1]{"#1"}

% Getting tired of writing \foreignTerm all the time
\newcommand{\farray}{\foreignTerm{array}\xspace}
\newcommand{\fArray}{\foreignTerm{Array}\xspace}
\newcommand{\foverhead}{\foreignTerm{overhead}\xspace}
\newcommand{\fOverhead}{\foreignTerm{Overhead}\xspace}
\newcommand{\fsubarray}{\foreignTerm{subarray}\xspace}
\newcommand{\fSubarray}{\foreignTerm{Subarray}\xspace}
\newcommand{\fbasecase}{\foreignTerm{base case}\xspace}
\newcommand{\fBasecase}{\foreignTerm{Base case}\xspace}
\newcommand{\ftopdown}{\foreignTerm{top down}\xspace}
\newcommand{\fTopdown}{\foreignTerm{Top down}\xspace}
\newcommand{\fbottomup}{\foreignTerm{bottom up}\xspace}
\newcommand{\fBottomup}{\foreignTerm{Bottom up}\xspace}
\newcommand{\fpruning}{\foreignTerm{pruning}\xspace}
\newcommand{\fPruning}{\foreignTerm{Pruning}\xspace}

\newcommand{\fgraph}{\foreignTerm{graph}\xspace}
\newcommand{\fGraph}{\foreignTerm{Graph}\xspace}
\newcommand{\fnode}{\foreignTerm{node}\xspace}
\newcommand{\fNode}{\foreignTerm{Node}\xspace}
\newcommand{\fedge}{\foreignTerm{edge}\xspace}
\newcommand{\fEdge}{\foreignTerm{Edge}\xspace}
\newcommand{\fdegree}{\foreignTerm{degree}\xspace}
\newcommand{\fDegree}{\foreignTerm{Degree}\xspace}
\newcommand{\fadjacencylist}{\foreignTerm{adjacency list}\xspace}
\newcommand{\fAdjacencylist}{\foreignTerm{Adjacency list}\xspace}
\newcommand{\fadjacencymatrix}{\foreignTerm{adjacency matrix}\xspace}
\newcommand{\fAdjacencymatrix}{\foreignTerm{Adjacency matrix}\xspace}
\newcommand{\fedgelist}{\foreignTerm{edge list}\xspace}
\newcommand{\fEdgelist}{\foreignTerm{Edge list}\xspace}
\newcommand{\flist}{\foreignTerm{list}\xspace}
\newcommand{\fList}{\foreignTerm{List}\xspace}
\newcommand{\fgraphtraversal}{\foreignTerm{graph traversal}\xspace}
\newcommand{\fGraphtraversal}{\foreignTerm{Graph traversal}\xspace}
\newcommand{\ftree}{\foreignTerm{tree}\xspace}
\newcommand{\fTree}{\foreignTerm{Tree}\xspace}
\newcommand{\fsubtree}{\foreignTerm{subtree}\xspace}
\newcommand{\fSubtree}{\foreignTerm{Subtree}\xspace}

\newcommand{\fDivideAndConquer}{\foreignTerm{Divide and Conquer}\xspace}
\newcommand{\fMergeSort}{\foreignTerm{Merge Sort}\xspace}
\newcommand{\fQuickSort}{\foreignTerm{Quicksort}\xspace}
\newcommand{\fpivot}{\foreignTerm{pivot}\xspace}
\newcommand{\fPivot}{\foreignTerm{Pivot}\xspace}
\newcommand{\fbruteForce}{\foreignTerm{brute force}\xspace}
\newcommand{\fBruteForce}{\foreignTerm{Brute force}\xspace}
\newcommand{\fCompleteSearch}{\foreignTerm{complete search}\xspace}
\newcommand{\fExhaustiveSearch}{\foreignTerm{exhaustive search}\xspace}
\newcommand{\fBinarySearch}{\foreignTerm{binary search}\xspace}
\newcommand{\fGreedy}{\foreignTerm{greedy}\xspace}
\newcommand{\fGreedyChoice}{\foreignTerm{greedy choice}\xspace}

\newcommand{\pheap}{\foreignTerm{heap}\xspace}
\newcommand{\pHeap}{\foreignTerm{Heap}\xspace}
\newcommand{\pBinaryHeap}{\foreignTerm{Binary Heap}\xspace}
\newcommand{\pbinaryHeap}{\foreignTerm{binary heap}\xspace}
\newcommand{\pHeapSort}{\foreignTerm{Heap Sort}\xspace}
\newcommand{\pdjs}{\foreignTerm{disjoint set}\xspace}
\newcommand{\pDjs}{\foreignTerm{Disjoint set}\xspace}
\usepackage{tikz}

\title{Matematika Diskret Dasar:\newline Kombinatorika}
\author{Tim Olimpiade Komputer Indonesia}
\date{}

\begin{document}

\begin{frame}
\titlepage
\end{frame}

\begin{frame}
\frametitle{Pendahuluan}
Melalui dokumen ini, kita akan:
\begin{itemize}
  \item Mempelajari \newTerm{aturan perkalian} dan \newTerm{aturan penjumlahan}.
  \item Mempelajari \newTerm{permutasi}.
  \item Mempelajari \newTerm{kombinasi}.
  \item Memahami \newTerm{segitiga pascal}.
\end{itemize}
\end{frame}

\section{Aturan Perkalian dan Aturan Penjumlahan}
\frame{\sectionpage}

\begin{frame}
\frametitle{Contoh Soal 1}
\begin{itemize}
  \item Terdapat 3 buah kota yaitu A, B, dan C.
  \item Kota A dan kota B terhubung oleh 3 jalur berbeda yaitu $e_{1}$, $e_{2}$, dan $e_{3}$.
  \item Sedangkan kota B dan kota C terhubung oleh 2 jalur berbeda yaitu $e_{4}$ dan $e_{5}$.
  \item Berapa banyak cara berbeda untuk menuju kota C dari kota A?
  \item Ilustrasi: 
  \newline
  \begin{tikzpicture}[auto, node distance=3cm, every loop/.style={},
                      thick,main node/.style={circle,draw,font=\sffamily\Large\bfseries}]
  
    \node[main node] (1) {A};
    \node[main node] (2) [right of=1] {B};
    \node[main node] (3) [right of=2] {C};
  
    \path[every node/.style={font=\sffamily\small}]
      (1) edge[bend left] node {$e_{1}$} (2)
      	  edge node {$e_{2}$} (2)
          edge[bend right] node {$e_{3}$} (2)
      (2) edge[bend left] node {$e_{4}$} (3)
          edge[bend right] node {$e_{5}$} (3);
  \end{tikzpicture}
\end{itemize}
\end{frame}

\begin{frame}
\frametitle{Solusi Awal}
\begin{itemize}
  \item Apabila kita hitung satu per satu, maka cara yang berbeda untuk menuju kota C dari kota A adalah sebagai berikut:
  \begin{itemize}
    \item Melalui jalur $e_{1}$ kemudian jalur $e_{4}$
    \item Melalui jalur $e_{1}$ kemudian jalur $e_{5}$
    \item Melalui jalur $e_{2}$ kemudian jalur $e_{4}$
    \item Melalui jalur $e_{2}$ kemudian jalur $e_{5}$
    \item Melalui jalur $e_{3}$ kemudian jalur $e_{4}$
    \item Melalui jalur $e_{3}$ kemudian jalur $e_{5}$
  \end{itemize}
  \item Dengan kata lain, terdapat 6 cara berbeda untuk menuju kota C dari kota A.
  \item Tetapi, apabila jumlah kota dan jalur yang ada sangatlah banyak, kita tidak mungkin menulis satu per satu cara yang berbeda. Karena itulah kita gunakan \newTerm{aturan perkalian}.
\end{itemize}
\end{frame} 

\begin{frame}
\frametitle{Aturan Perkalian}
\begin{itemize}
  \item Misalkan suatu proses dapat dibagi menjadi $N$ subproses independen yang mana terdapat $a_{i}$ cara untuk menyelesaikan subproses ke-$i$.
  \item Banyak cara yang berbeda untuk menyelesaikan proses tersebut adalah $a_{1} \times a_{2} \times a_{3} \times ... \times a_{i}$.	
\end{itemize}
\end{frame}

\begin{frame}
\frametitle{Solusi dengan Aturan Perkalian}
\begin{itemize}
  \item Anggaplah bahwa perjalanan dari kota A menuju kota B merupakan subproses pertama, yang mana terdapat 3 cara untuk menyelesaikan subproses tersebut.
  \item Anggaplah pula bahwa perjalanan dari kota B menuju kota C merupakan subproses kedua, yang mana terdapat 2 cara untuk menyelesaikan subproses tersebut.
  \item Karena perjalanan dari kota A menuju kota B dan dari kota B menuju kota C merupakan 2 subproses yang berbeda, maka kita dapat menggunakan aturan perkalian.
  \item Banyak cara berbeda dari kota A menuju kota C adalah $3 \times 2 = 6$.
\end{itemize}
\end{frame}

\begin{frame}
\frametitle{Contoh Soal 2}
\begin{itemize}
  \item Contoh soal ini merupakan lanjutan dari \newTerm{Contoh Soal 1}.
  \item Deskripsi soal, jumlah kota dan jalur serta susunan jalur yang ada sama persis dengan soal tersebut.
  \item Tambahkan 1 jalur lagi, yaitu $e_{6}$ yang menghubungkan kota A dan C.
  \item Berapa banyak cara berbeda untuk menuju kota C dari kota A?
  \item Ilustrasi:
  \newline
  \begin{tikzpicture}[auto, node distance=3cm, every loop/.style={},
                        thick,main node/.style={circle,draw,font=\sffamily\Large\bfseries}]
    
      \node[main node] (1) {A};
      \node[main node] (2) [right of=1] {B};
      \node[main node] (3) [right of=2] {C};
    
      \path[every node/.style={font=\sffamily\small}]
        (1) edge[bend left] node {$e_{1}$} (2)
        	edge node {$e_{2}$} (2)
            edge[bend right] node {$e_{3}$} (2)
        (2) edge[bend left] node {$e_{4}$} (3)
            edge[bend right] node {$e_{5}$} (3)
        (3) edge[bend left] node {$e_{6}$} (1);
    \end{tikzpicture}
\end{itemize}
\end{frame}

\begin{frame}
\frametitle{Analisis Contoh Soal}
\begin{itemize}
  \item Dengan mencoba satu persatu setiap cara, maka terdapat 7 cara yang berbeda, yaitu 6 cara sesuai dengan soal sebelumnya, ditambah dengan menggunakan jalur $e_{6}$.
  \item Apabila kita menggunakan aturan perkalian, maka didapatkan banyak cara yang berbeda adalah $3 \times 2 \times 1 = 6$ yang mana jawaban tersebut tidaklah tepat.
  \item Kita tidak dapat menggunakan aturan perkalian dalam permasalahan ini, karena antara perjalanan dari kota A menuju kota C melalui kota B dengan tanpa melalui kota B merupakan 2 proses yang berbeda.
  \item Oleh karena itu, kita dapat menggunakan \newTerm{aturan penjumlahan}.
\end{itemize}
\end{frame}

\begin{frame}
\frametitle{Aturan Penjumlahan}
\begin{itemize}
  \item Misalkan suatu proses dapat dibagi menjadi $N$ himpunan proses berbeda yaitu $H_{1}, H_{2}, H_{3}, ... , H_{N}$ dengan setiap himpunannya saling lepas (tidak beririsan).
  \item Banyak cara yang berbeda untuk menyelesaikan proses tersebut adalah $|H_{1}| + |H_{2}| + |H_{3}| + ... + |H_{N}|$ dengan $|H_{i}|$ merupakan banyaknya cara berbeda untuk menyelesaikan proses ke-$i$.
\end{itemize}
\end{frame}

\begin{frame}
\frametitle{Solusi dengan Aturan Penjumlahan}
\begin{itemize}
  \item Proses perjalanan dari kota A menuju kota C dapat kita bagi menjadi 2 himpunan proses yang berbeda, yaitu dari kota A menuju kota C melalui kota B, dan dari kota A langsung menuju kota C.
  \item Banyak cara dari kota A menuju kota C melalui kota B dapat kita dapatkan dengan aturan perkalian seperti yang dibahas pada permasalahan sebelumnya, yaitu 6 cara berbeda.
  \item Banyak cara dari kota A langsung menuju kota C adalah 1 cara, yaitu melalui jalur $e_{6}$.
  \item Dengan aturan penjumlahan, banyak cara berbeda dari kota A menuju kota C adalah $6 + 1 = 7$ cara berbeda.
\end{itemize}
\end{frame}

\begin{frame}
\frametitle{Hati-Hati!}
\begin{itemize}
  \item Apabila terdapat irisan dari himpunan proses tersebut, maka solusi yang kita dapatkan dengan aturan penjumlahan menjadi tidak tepat, karena ada solusi yang terhitung lebih dari sekali.
  \item Agar solusi tersebut menjadi tepat, gunakan \textcolor{blue}{\href{https://en.wikipedia.org/wiki/Inclusion-exclusion\_principle}{Prinsip Inklusi-Eksklusi}} pada \textcolor{blue}{\href{https://id.wikipedia.org/wiki/Himpunan}{Teori Himpunan}}.
\end{itemize}
\end{frame}

\section{Permutasi}
\frame{\sectionpage}

\begin{frame}
\frametitle{Perkenalan Permutasi}
\begin{itemize}
  \item Permutasi adalah pemilihan urutan beberapa elemen dari suatu himpunan.
  \item Untuk menyelesaikan soal-soal permutasi, dibutuhkan pemahaman konsep faktorial.
\end{itemize}
\end{frame}

\begin{frame}
\frametitle{Notasi Faktorial}
\begin{itemize}
  \item Faktorial dari N ($N!$) merupakan hasil perkalian dari semua bilangan asli kurang dari atau sama dengan N.
  \item $N! = N \times (N-1) \times (N-2) \times ... \times 3 \times 2 \times 1$, dengan $0! = 1$.
  \item Contoh: $5! = 5 \times 4 \times 3 \times 2 \times 1 = 120$.
\end{itemize}
\end{frame}

\begin{frame}
\frametitle{Redundansi}
\begin{itemize}
  \item Redundansi sering disebut juga dengan \newTerm{aturan pembagian}.
  \item Apabila terdapat $K$ susunan cara berbeda yang kita anggap merupakan 1 cara yang sama, maka kita dapat membagi total keseluruhan cara dengan $K$, sehingga $K$ cara tersebut dianggap sama sebagai 1 cara.
\end{itemize}
\end{frame}

\begin{frame}
\frametitle{Contoh Redundansi}
\begin{itemize}
  \item Banyak kata berbeda yang disusun dari huruf-huruf penyusun "TOKI" adalah $4!$ (menggunakan aturan perkalian).
  \item Apabila kita ganti soal tersebut, yaitu kata berbeda yang disusun dari huruf-huruf penyusun "BACA", solusi $4!$ merupakan solusi yang salah.
  \item Sebab, terdapat 2 buah huruf 'A'. Sebagai contoh: $BA_{1}CA_{2}$ dan $BA_{2}CA_{1}$ pada dasarnya merupakan kata yang sama.
\end{itemize}
\end{frame}

\begin{frame}
\frametitle{Contoh Redundansi (lanj.)}
\begin{itemize}
  \item Terdapat 2! cara berbeda tetapi yang kita anggap sama, yaitu penggunaan $A_{1}A_{2}$ dan $A_{2}A_{1}$.
  \item Sehingga banyak kata berbeda yang dapat kita bentuk dari huruf-huruf penyusun kata "BACA" adalah $\frac{4!}{2!} = \frac{24}{2} = 12$ kata berbeda.
\end{itemize}
\end{frame}

\begin{frame}
\frametitle{Contoh Soal 1}
\begin{itemize}
  \item Terdapat 5 anak (sebut saja A, B, C, D, dan E) yang sedang mengikuti sebuah kompetisi.
  \item Dalam kompetisi tersebut akan diambil 3 peserta sebagai pemenang.
  \item Berapa banyak susunan pemenang yang berbeda dari kelima orang tersebut?
\end{itemize}
\end{frame}

\begin{frame}
\frametitle{Solusi Awal}
\begin{itemize}
  \item Anggap bahwa kita mengambil semua anak sebagai pemenang, sehingga terdapat $5! = 120$ susunan pemenang yang berbeda (ABCDE, ABCED, ABDCE, ..., EDCBA).
  \item Apabila kita hanya mengambil 3 peserta saja, perhatikan bahwa terdapat 2 cara berbeda yang kita anggap sama. Contoh: (ABC)DE dan (ABC)ED merupakan cara yang sama, karena 3 peserta yang menang adalah A, B, dan C.
  \item Dengan menggunakan aturan pembagian, maka banyak susunan pemenang yang berbeda adalah $\frac{5!}{2!} = \frac{120}{2} = 60$ susunan berbeda.
\end{itemize}
\end{frame}

\begin{frame}
\frametitle{Solusi Secara Umum}
\begin{itemize}
  \item Apabila terdapat N anak, dan kita mengambil semua anak sebagai pemenang, maka terdapat $N!$ susunan cara berbeda.
  \item Tetapi apabila kita hanya mengambil R anak saja, maka akan terdapat $(N-R)!$ susunan berbeda yang kita anggap sama.
  \item Secara umum dengan aturan pembagian, banyak susunan berbeda adalah $\frac{N!}{(N-R)!}$.
  \item Inilah yang kita kenal dengan istilah \newTerm{permutasi}.
\end{itemize}
\end{frame}


\begin{frame}
\frametitle{Permutasi}
\begin{itemize}
  \item Misalkan terdapat $n$ objek dan kita akan mengambil $r$ objek dari n objek tersebut yang mana $r<n$ dan urutan pengambilan diperhitungkan.
  \item Banyak cara pengambilan yang berbeda adalah permutasi $r$ terhadap $n$: 
  \item $P(n,r) = _{n}P_{r} = P^{n}_{r} = \frac{n!}{(n-r)!}$.
\end{itemize}
\end{frame}

\begin{frame}
\frametitle{Contoh Soal 2}
\begin{itemize}
  \item Contoh soal ini sejenis dengan contoh pada aturan pembagian.
  \item Berapa banyak kata berbeda yang disusun dari huruf-huruf penyusun kata "MEGAGIGA"?
\end{itemize}
\end{frame}

\begin{frame}
\frametitle{Solusi Awal}
\begin{itemize}
  \item Terdapat 8 huruf, sehingga banyak kata yang dapat disusun adalah $8!$.
  \item Terdapat 3 huruf 'G' sehingga terdapat 6 kata berbeda yang kita anggap sama ($G_{1}G_{2}G_{3}, G_{1}G_{3}G_{2}, ...,G_{3}G_{2}G_{1}$).
  \item Dengan aturan pembagian, maka banyak kata yang dapat disusun mengingat kesamaan kata pada huruf G adalah $\frac{8!}{3!}$.
  \item Perlu kita perhatikan pula bahwa terdapat 2 huruf A, sehingga dengan cara yang sama akan didapatkan banyak kata yang berbeda adalah $\frac{8!}{(3! \times 2!)}$.
\end{itemize}
\end{frame}

\begin{frame}
\frametitle{Solusi Secara Umum}
\begin{itemize}
  \item Terdapat N huruf, sehingga banyak kata yang dapat kita susun adalah $N!$.
  \item Apabila terdapat K huruf dengan setiap hurufnya memiliki $R_{i}$ huruf yang sama, maka dengan aturan pembagian banyak kata berbeda yang dapat disusun adalah $\frac{N!}{(R_{1}! \times R_{2}! \times R_{3}! \times ... \times R_{K}!)}$.
  \item Inilah yang kita kenal dengan \newTerm{permutasi elemen berulang}.
\end{itemize}
\end{frame}

\begin{frame}
\frametitle{Permutasi Elemen Berulang}
\begin{itemize}
  \item Misalkan terdapat $n$ objek dan terdapat $k$ objek yang mana setiap objeknya memiliki $r_{i}$ elemen yang berulang, maka banyaknya cara berbeda dalam menyusun objek tersebut adalah:
  \item $P^{n}_{r_{1},r_{2},r_{3},...,r_{k}} = \frac{n!}{(r_{1}! \times r_{2}! \times r_{3}! \times ... \times r_{k}!)}$ 
\end{itemize}
\end{frame}

\begin{frame}
\frametitle{Contoh Soal 3}
\begin{itemize}
  \item Terdapat 4 anak, sebut saja A, B, C, dan D.
  \item Berapa banyak susunan posisi duduk yang berbeda apabila mereka duduk melingkar?
\end{itemize}
\end{frame}

\begin{frame}
\frametitle{Solusi Awal}
\begin{itemize}
  \item Banyak susunan posisi duduk yang berbeda apabila mereka duduk seperti biasa (tidak melingkar) adalah $4! = 120$.
  \item Perhatikan bahwa posisi duduk ABCD, BCDA, CDAB, dan DABC merupakan susunan yang sama apabila mereka duduk melingkar, karena susunan tersebut merupakan rotasi dari susunan yang lainnya.
  \item Dengan kata lain terdapat 4 cara berbeda yang kita anggap sama.
  \item Dengan aturan pembagian, banyak susunan posisi duduk yang berbeda adalah $\frac{120}{4} = 30$ susunan berbeda.
\end{itemize}
\end{frame}


\begin{frame}
\frametitle{Solusi Secara Umum}
\begin{itemize}
  \item Banyak susunan posisi duduk yang berbeda apabila $N$ anak seperti biasa (tidak melingkar) adalah $N!$.
  \item Dengan analisis yang sama, akan terdapat $N$ cara berbeda yang kita anggap sama.
  \item Dengan aturan pembagian, banyak susunan posisi duduk yang berbeda adalah $\frac{N!}{N} = (N-1)!$ susunan berbeda.
  \item inilah yang kita kenal dengan istilah \newTerm{permutasi siklis}.
\end{itemize}
\end{frame}

\begin{frame}
\frametitle{Permutasi Siklis}
\begin{itemize}
  \item Permutasi siklis adalah permutasi yang disusun melingkar.
  \item Banyaknya susunan yang berbeda dari permutasi siklis terhadap $n$ objek adalah:
  \item $P^{n}_{(siklis)} = (n-1)!$
\end{itemize}
\end{frame}

\section{Kombinasi}
\frame{\sectionpage}

\begin{frame}
\frametitle{Contoh Soal 1}
\begin{itemize}
  \item Terdapat 5 anak (sebut saja A, B, C, D, dan E) yang mana akan dipilih 3 anak untuk mengikuti kompetisi.
  \item Berapa banyak susunan tim berbeda yang dapat dibentuk?
\end{itemize}
\end{frame}

\begin{frame}
\frametitle{Solusi Awal}
\begin{itemize}
  \item Soal ini berbeda dengan permutasi, karena susunan ABC dan BAC merupakan susunan yang sama, yaitu 1 tim terdiri dari A, B, dan C.
  \item Apabila kita anggap bahwa mereka merupakan susunan yang berbeda, maka banyaknya susunan tim adalah $P^{5}_{2} = \frac{120}{2} = 60$.
  \item Untuk setiap susunan yang terdiri dari anggota yang sama akan terhitung 6 susunan berbeda yang mana seharusnya hanya dihitung sebagai 1 susunan yang sama.
  \item Contoh: ABC, ACB, BAC, BCA, CAB, CBA merupakan 1 susunan yang sama.
  \item Oleh karena itu dengan aturan pembagian, banyaknya susunan tim yang berbeda adalah $\frac{60}{6} = 10$ susunan berbeda.
\end{itemize}
\end{frame}

\begin{frame}
\frametitle{Solusi Secara Umum}
\begin{itemize}
  \item Misalkan terdapat $N$ anak dan akan kita ambil $R$ anak untuk dibentuk sebagai 1 tim.
  \item Apabila urutan susunan diperhitungkan, maka banyaknya susunan tim adalah $P^{N}_{R}$.
  \item Setiap susunan yang terdiri dari anggota yang sama akan terhitung $R!$ susunan berbeda yang mana seharusnya hanya dihitung sebagai 1 susunan yang sama.
  \item Dengan aturan pembagian, banyaknya susunan tim yang berbeda adalah $\frac{P^{N}_{R}}{R!} = \frac{N!}{(N-R)! \times R!}$ susunan berbeda.
  \item Inilah yang kita kenal dengan istilah \newTerm{kombinasi}.
\end{itemize}
\end{frame}

\begin{frame}
\frametitle{Kombinasi}
\begin{itemize}
  \item Misalkan terdapat $n$ objek dan kita akan mengambil $r$ objek dari $n$ objek tersebut dengan $r<n$ dan urutan pengambilan tidak diperhitungkan.
  \item Banyaknya cara pengambilan yang berbeda adalah kombinasi $r$ terhadap $n$:
  \item $C(n,r) = _{n}C_{r} = C^{n}_{r} = \frac{n!}{(n-r)! \times r!}$.
\end{itemize}
\end{frame}

\begin{frame}
\frametitle{Contoh Soal 2}
\begin{itemize}
  \item Pak Dengklek ingin membeli kue pada toko kue yang menjual 3 jenis kue, yaitu rasa coklat, stroberi, dan kopi.
  \item Apabila Pak Dengklek ingin membeli 4 buah kue, maka berapa banyak kombinasi kue berbeda yang Pak Dengklek dapat beli?
\end{itemize}
\end{frame}

\begin{frame}
\frametitle{Analisis Contoh Soal}
\begin{itemize}
  \item Perhatikan bahwa membeli coklat-stroberi dengan stroberi-coklat akan menghasilkan kombinasi yang sama.
  \item Contoh soal ini merupakan permasalahan kombinasi.
  \item Akan tetapi, kita dapat membeli suatu jenis kue beberapa kali atau bahkan tidak membeli sama sekali.
  \item Contoh soal ini dapat dimodelkan secara matematis menjadi mencari banyaknya kemungkinan nilai A, B, dan C yang memenuhi $A + B + C = 4$ dan $A,B,C \geq 0$.
\end{itemize}
\end{frame}

\begin{frame}
\frametitle{Solusi}
\begin{itemize}
  \item Kita dapat membagi 4 kue tersebut menjadi 3 bagian. Untuk mempermudah ilustrasi tersebut, kita gunakan lambang \texttt{o} yang berarti kue, dan \texttt{|} yang berarti pembatas.
  \item Bagian kiri merupakan kue A, bagian tengah merupakan kue B, dan bagian kanan merupakan kue C.
  \item Contoh: \texttt{(o|o|oo)} menyatakan 1 kue A, 1 kue B, dan 2 kue C.
  \item Contoh lain: \texttt{(oo|oo|)} menyatakan 2 kue A, 2 kue B, dan 0 kue C.
  \item Dengan kata lain, semua susunan yang mungkin adalah \texttt{(oooo||)}, \texttt{(ooo|o|)}, \texttt{(oo|oo|)}, ..., \texttt{(||oooo)} yang tidak lain merupakan $C^{6}_{2} = \frac{6!}{4! \times 2!} = 15$ susunan berbeda.
\end{itemize}
\end{frame}

\begin{frame}
\frametitle{Solusi Secara Umum}
\begin{itemize}
  \item Kita ingin mencari banyaknya susunan nilai berbeda dari $X_{1}, X_{2}, X_{3}, ..., X_{r}$ yang mana $X_{1} + X_{2} + X_{3} + ... + X_{r} = n$.
  \item Untuk membagi $n$ objek tersebut menjadi $r$ bagian, maka akan dibutuhkan $r-1$ buah pembatas, sehingga akan terdapat $n+r-1$ buah objek, yang mana kita akan memilih $r-1$ objek untuk menjadi simbol \texttt{|}.
  \item Dengan kata lain, banyaknya susunan nilai yang berbeda adalah $C^{n+r-1}_{r-1}$.
  \item Inilah yang kita kenal dengan istilah \newTerm{kombinasi dengan pengulangan}.
\end{itemize}
\end{frame}

\begin{frame}
\frametitle{Kombinasi dengan Pengulangan}
\begin{itemize}
  \item Misalkan terdapat $r$ jenis objek dan kita akan mengambil $n$ objek, dengan tiap jenisnya dapat diambil 0 atau beberapa kali.
  \item Banyaknya cara berbeda yang memenuhi syarat tersebut adalah sebagai berikut:
  \item $C^{n+r-1}_{n} = C^{n+r-1}_{r-1} = \frac{(n+r-1)!}{n! \times (r-1)!}$
\end{itemize}
\end{frame}

\section{Segitiga Pascal}
\frame{\sectionpage}

\begin{frame}
\frametitle{Segitiga Pascal}
\begin{itemize}
  \item Segitiga Pascal merupakan susunan dari koefisien-koefisien binomial dalam bentuk segitiga.
  \item Nilai dari baris ke-$n$ suku ke-$r$ adalah $C^{n}_{r}$.
  \item Contoh Segitiga Pascal:
  \begin{itemize} 
    \item Baris ke-1: 1
    \item Baris ke-2: 1 1
    \item Baris ke-3: 1 2 1
    \item Baris ke-4: 1 3 3 1
    \item Baris ke-5: 1 4 6 4 1
  \end{itemize}
\end{itemize}
\end{frame}

\begin{frame}
\frametitle{Analisis}
\begin{itemize}
  \item Diberikan suatu himpunan $S = \{X_{1},X_{2},...,X_{n}\}$. Berapa banyak cara untuk memilih $r$ objek dari $S$?
  \item Terdapat 2 kasus:
  \begin{itemize}
    \item Kasus 1: $X_{n}$ dipilih.
    Artinya, $r-1$ objek harus dipilih dari himpunan $\{X_{1},X_{2},X_{3},...,X_{n-1}\}$. Banyaknya cara berbeda dari kasus ini adalah $C^{n-1}_{r-1}$.
    \item Kasus 2: $X_{n}$ tidak dipilih. Artinya, $r$ objek harus dipilih dari himpunan $\{X_{1},X_{2},X_{3},...,X_{n}\}$. Banyaknya cara berbeda dari kasus ini adalah $C^{n-1}_{r}$.
  \end{itemize}
\end{itemize}
\end{frame}

\begin{frame}
\frametitle{Analisis (lanj.)}
\begin{itemize}
  \item Dengan aturan penjumlahan dari kasus 1 dan kasus 2, kita dapatkan $C^{n}_{r} = C^{n-1}_{r-1} + C^{n-1}_{r}$.
  \item Persamaan itulah yang sering kita gunakan dalam membuat Segitiga Pascal, karena $C^n_r$ juga menyatakan baris ke-$n$ dan kolom ke-$r$ pada Segitiga Pascal.
\end{itemize}
\end{frame}

\begin{frame}
\frametitle{Analisis (lanj.)}
\begin{itemize}
  \item Dalam dunia pemrograman, kadang kala dibutuhkan perhitungan seluruh nilai $C^{n}_{r}$ yang memenuhi $n \le N$, untuk suatu $N$ tertentu.
  \item Pencarian nilai dari $C^{n}_{r}$ dengan menghitung faktorial memiliki kompleksitas $O(n)$.
  \item Apabila seluruh nilai kombinasi dicari dengan cara tersebut, kompleksitas akhirnya adalah $O(N^3)$.
  \item Namun, dengan menggunakan persamaan $C^{n}_{r} = C^{n-1}_{r-1} + C^{n-1}_{r}$, maka secara keseluruhan kompleksitasnya dapat menjadi $O(N^{2})$.
\end{itemize}
\end{frame}

\begin{frame}[fragile]
\frametitle{Implementasi Segitiga Pascal (lanj.)}
\begin{codebox}
\Procname{$\proc{segitiga\_pascal}(N)$}
\li \Comment Sediakan array 2-dimensi $C$ berukuran $(N+1) \times (N+1)$
\li \For $i \gets 0$ \To $N$
    \Do
\li   $C[i][0] \gets 1$
\li   \For $j \gets 0$ \To $i-1$
      \Do
\li     $C[i][j] = C[i-1][j-1] + C[i-1][j]$
      \End
\li   $C[i][i] \gets 1$
    \End
\end{codebox}
\end{frame}

\begin{frame}
\frametitle{Penggunaan Segitiga Pascal}
\begin{itemize}
  \item Dalam bidang matematika, Segitiga Pascal merupakan kumpulan dari koefisien binomial yang dapat digunakan dalam Binomial Newton $(x+y)^{n} = \sum\limits_{r=0}^{n} a_{r} x^{n-r} y^{r}$ yang mana $a_{r}$ merupakan bilangan dalam segitiga pascal baris ke-$n$ suku ke-$r$.
  \item Dalam bidang programming, algoritma dari segitiga pascal dapat digunakan untuk mencari semua nilai dari kombinasi $r$ terhadap $n$ untuk seluruh $n \le N$ dengan kompleksitas waktu $O(N^{2})$ dan memori $O(N^{2})$.
\end{itemize}
\end{frame}

\begin{frame}
\frametitle{Penutup}
\begin{itemize}
  \item Materi ini berisi mengenai Kombinatorika Dasar dan implementasinya yang umum digunakan dalam pemrograman kompetitif.
  \item Dengan materi ini, Anda diharapkan dapat menggunakan konsep kombinatorika yang ada.
\end{itemize}
\end{frame}

\end{document}
